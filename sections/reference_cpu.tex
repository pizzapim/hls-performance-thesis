In this chapter, we present our reference CPU application, as well as an experimental setup for performance analysis.

\section{Reference implementation} \label{section:ref_impl}

To analyze the performance of an FPGA implementation, we have programmed a CPU-based reference implementation in the C language.
We have chosen this language for its performance and for easy portability to an FPGA kernel in C, C++ or OpenCL.
As will be explained in section \ref{section:experiment_setup}, we will analyze the performance of string-matching.
Therefore we can precalculate all auxiliary data structures that were previously explained in section \ref{section:background_stringmatching}.

\subsection{Burrows-Wheeler transform construction} \label{section:bwt_construction}

The method to obtain the Burrows-Wheeler transform $\Tbw$ from a text can however not be used in practice.
We previously explained in section \ref{section:background_bwt} that, to obtain $\Tbw$, we list all rotations of the text and then lexicographically sort them.
This would unfortunately load to a space complexity of $O(n^2)$ which is infeasible for large data sets.
To avoid this issue, we can derive $\Tbw$ from the suffix array of the text \cite{ullah_implementation_2020}.
The suffix array is essentially a Burrows-Wheeler transform, but holds indices instead of characters.
Deriving $\Tbw$ is therefore as easy as indexing the text for each index in the suffix array.

The C code for obtaining the suffix array from a text is shown in listing \ref{listing:construct_sa}.
First we simply list all indices of the text.
Then we use quicksort to sort the indices: for each pair of indices, the strings starting from these positions are lexicographically compaired.

\begin{listing}[ht]
\begin{minted}{c}
sa_t *ConstructSuffixArray(char *text, size_t text_sz) {
  sa_t *suffix_array = calloc(text_sz + 1, sizeof(sa_t));

  for (size_t i = 0; i < text_sz + 1; ++i)
    suffix_array[i] = i;

  qsort_r(suffix_array, text_sz + 1, sizeof(sa_t), &CompareSuffixArray, text);

  return suffix_array;
}
\end{minted}
\caption{Generating the suffix array for a text.}
\label{listing:construct_sa}
\end{listing}

\subsection{Space reduction} \label{section:space_reduction}

Another concern is the storage usage of the resulting index.
Consider an ASCII text file of 3 gigabyte with alphabet size 27 (meaning 27 distinct characters).
The reference implementation uses an \mintinline{c}{unsigned int} to store the numbers in the rank matrix and the suffix array, which takes up two bytes in the testing environment.
The resulting storage usage of the index is divided as follows:

\begin{itemize}
  \item \textbf{Burrows-Wheeler transform}: \SI{3}{GB}
  \item \textbf{Rank matrix}: $2 \cdot 27 \cdot \SI{3}{GB} = \SI{162}{GB}$
  \item \textbf{Suffix array}: $2 \cdot \SI{3}{GB} = \SI{6}{GB}$
  \item \textbf{Resulting storage usage}: $3 + 162 + \SI{6}{GB} = \SI{171}{GB}$
\end{itemize}

It is clear that that the resulting \SI{171}{GB} of storage usage is unacceptable when it is loaded into memory.
Therefore, we should reduce the storage usage of the rank matrix and suffix array.
Our reference implementation only stores every $n$ rows of the rank matrix and suffix array \cite{fernandez_string_2011,langmead_introduction_nodate}.
The rank matrix and suffix array can then be reduced as needed.
At run time, the missing rows can be calculated on the fly, which can be done in constant time.

[TODO: explain how we reduce the number of rows?]

\section{Experimental setup} \label{section:experiment_setup}
We investigate the performance of the FPGA implementation based on two metrics found in literature.
To evaluate the overall performance, throughput is measured by counting how many patterns can be matched (finding their indices) per second.
As one of the main benefits of FPGAs is energy efficiency, we also evaluate this by measuring the average energy usage for matching one pattern.

As test data, we use several corpora from the Pizza \& Chili corpus collection \cite{ferragina_pizzachili_nodate}, preprocessed on the CPU.
This collection is often used for text compression and indexing, and features corpora from multiple sources, such as protein sequences, program source codes and DNA sequences \cite{makinen_compressed_2007}.
For each corpus we also evaluate a few text sizes.

As input data, we take an amount of substrings found in the original string that are matched using the preprocessed FM-index.
Different lengths of the patterns are be investigated.
The amount of patterns is fixed, as we are interested in the throughput and average energy consumption.
However, the amount of patterns should be big enough to exploit parallelization on the FPGA.

For big texts, we will use the technique described in \ref{section:bwt_construction} to reduce the size of the rank matrix and suffix array.

In terms of hardware we will use the Xilinx U250 Data Center Accelerator Card as the FPGA.
The CPU is to be determined.

We expect that the FPGA implementation will have a higher throughput, as there is potential for high paralellization.
We also expect the FPGA implementation to be more energy efficient, as is usually the case.

[TODO: show which corpora we actually use and information about them]

[TODO: explain comparison]

[TODO: CPU hardware summary]


\section{Results}
