In this chapter, we present our FPGA implementation of the FM-index algorithm, and present an experimental setup to compare performance against a reference implemenation.

\section{Reference implementation} \label{section:ref_impl}

To analyze the performance of an FPGA implementation, we have programmed a CPU-based reference implementation in the C language.
We have chosen this language for its performance and for easy portability to an FPGA kernel in C, C++ or OpenCL.
As will be explained in section \ref{section:experiment_setup}, we will analyze the performance of string-matching.
Therefore we can precalculate all auxiliary data structures that were previously explained in section \ref{section:background_stringmatching}.

\subsection{Burrows-Wheeler transform construction} \label{section:bwt_construction}

The method to obtain the Burrows-Wheeler transform $\Tbw$ from a text can however not be used in practice.
We previously explained in section \ref{section:background_bwt} that, to obtain $\Tbw$, we list all rotations of the text and then lexicographically sort them.
This would unfortunately load to a space complexity of $O(n^2)$ which is infeasible for large data sets.
To avoid this issue, we can derive $\Tbw$ from the suffix array of the text \cite{ullah_implementation_2020}.
The suffix array is essentially a Burrows-Wheeler transform, but holds indices instead of characters.
Deriving $\Tbw$ is therefore as easy as indexing the text for each index in the suffix array.

The C code for obtaining the suffix array from a text is shown in listing \ref{listing:construct_sa}.
First we simply list all indices of the text.
Then we use quicksort to sort the indices: for each pair of indices, the strings starting from these positions are lexicographically compaired.

\begin{listing}[ht]
\begin{minted}{c}
sa_t *ConstructSuffixArray(char *text, size_t text_sz) {
  sa_t *suffix_array = calloc(text_sz + 1, sizeof(sa_t));

  for (size_t i = 0; i < text_sz + 1; ++i)
    suffix_array[i] = i;

  qsort_r(suffix_array, text_sz + 1, sizeof(sa_t), &CompareSuffixArray, text);

  return suffix_array;
}
\end{minted}
\caption{Generating the suffix array for a text.}
\label{listing:construct_sa}
\end{listing}

\subsection{Space reduction} \label{section:space_reduction}

Another concern is the storage usage of the resulting index.
Consider an ASCII text file of 3 gigabyte with alphabet size 27 (meaning 27 distinct characters).
The reference implementation uses an \mintinline{c}{unsigned int} to store the numbers in the rank matrix and the suffix array, which takes up two bytes in the testing environment.
The resulting storage usage of the index is divided as follows:

\begin{itemize}
  \item \textbf{Burrows-Wheeler transform}: \SI{3}{GB}
  \item \textbf{Rank matrix}: $2 \cdot 27 \cdot \SI{3}{GB} = \SI{162}{GB}$
  \item \textbf{Suffix array}: $2 \cdot \SI{3}{GB} = \SI{6}{GB}$
  \item \textbf{Resulting storage usage}: $3 + 162 + \SI{6}{GB} = \SI{171}{GB}$
\end{itemize}

It is clear that that the resulting \SI{171}{GB} of storage usage is unaccaptable when it is loaded into memory.
Therefore, we should reduce the storage usage of the rank matrix and suffix array.
Our reference implemention only stores every $n$ rows of the rank matrix and suffix array \cite{fernandez_string_2011,langmead_introduction_nodate}.
The rank matrix and suffix array can then be reduced as needed.
At run time, the missing rows can be calculated on the fly, which can be done in constant time.

[TODO: explain how we reduce the number of rows?]



\section{FPGA implementation} \label{section:fpga_impl}

[TODO: explain how we adapted the reference implementation to OpenCL]
