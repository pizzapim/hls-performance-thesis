\section{Experimental setup \& hypothesis} \label{section:experiment_setup_hypo}

\subsection{Experimental setup} \label{section:experiment_setup}

We investigate the performance on the CPU and FPGA using two metrics found in literature.
To evaluate the overall performance, throughput is measured by counting the average amount of patterns matched (finding the corresponding original positions) per second.
Additionally, to provide a more fair comparison between CPU and FPGA, the throughput is also expressed as the amount of patterns matched per clock cycle.
As one of the main benefits of FPGAs is energy efficiency, we also evaluate average energy usage for matching one pattern.

The string searches are performed on FM-indices which we preprocess on the CPU.
We use several corpora provided by the Pizza \& Chili corpus collection \cite{ferragina_pizzachili_nodate} to compute the FM-indices.
This collection is often used for text compression and indexing, and features corpora from a diverse set of sources \cite{makinen_compressed_2007}.
We use the \textit{dna}, \textit{proteins} and \textit{dblp.xml} corpora for our experiments.
An explanation for each corpus and the alphabet size is shown in table \ref{table:corpora}.

\begin{table}[H]
  \begin{tabularx}{\linewidth}{l X l}
    \hline
    \textbf{Corpus} & \textbf{Description} & \textbf{$|\Sigma|$} \\ \hline
    \textit{dna} & DNA sequences provided by Project Gutenberg. The DNA bases are encoded as A, C, G and T, but also contains special characters. & 16 \\ \hline
    \textit{proteins} & Protein sequences provided by the Swiss-Prot database. The 20 amino acids are encoded as upper-case letters. Also contains special characters. & 27 \\ \hline
    \textit{dblp.xml} & Bibliographic information on computer science journals in XML format, provided by DBLP. & 97 \\ \hline
  \end{tabularx}
  \caption{Description and alphabet size of the three corpora used from the Pizza \& Chili corpus collection.}
  \label{table:corpora}
\end{table}

To measure throughput, we generate as input for the experiment 7500 random patterns that occur in the text.
Generating random patterns from the alphabet alone gives too many patterns that have no occurences in the original text, especially for the \textit{dblp.xml} corpus which has a high alphabet size.
We therefore think taking patterns that actually occur in the original text better reflects a normal string search.
We experiment with different pattern lengths, namely lengths 4, 6 and 8.
We chose these lengths because smaller lengths would give way too many matches for corpora with a small alphabet size, while bigger lengths would give almost no matches at all for a large alphabet size.

We also investigate the effect of the size of the corpus on the throughput.
Therefore, for each corpus, we perform experiments on 10, 15 and 20 MB versions of the corpus.

Lastly, each experiment is performed five times to limit variability.

The CPU experiments are performed on an Intel Xeon Gold 6128 processor \cite{noauthor_intel_nodate}.
This CPU has a base frequency of 3.40 GHz.

\subsection{Hypothesis} \label{section:cpu_hypothesis}

We expect that a higher pattern length will result in a lower throughput, as string searching using the FM-index loops over the characters of the pattern.
Subsequently, we expect that the size of the corpus has a negligible effect on the throughput.
Lastly, we hypothesize that corpora with a larger alphabet, i.e. the \textit{dblp.xml} corpus, have a lower throughput, as the data structures are larger which could negatively impact memory performance.
