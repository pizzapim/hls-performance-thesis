\section{Experimental setup} \label{section:experiment_setup}
We investigate the performance of the FPGA implementation based on two metrics found in literature.
To evaluate the overall performance, throughput is measured by counting how many patterns can be matched (finding their indices) per second.
As one of the main benefits of FPGAs is energy efficiency, we also evaluate this by measuring the average energy usage for matching one pattern.

As test data, we use several corpora from the Pizza \& Chili corpus collection \cite{ferragina_pizzachili_nodate}, preprocessed on the CPU.
This collection is often used for text compression and indexing, and features corpora from multiple sources, such as protein sequences, program source codes and DNA sequences \cite{makinen_compressed_2007}.
For each corpus we also evaluate a few text sizes.

As input data, we take an amount of substrings found in the original string that are matched using the preprocessed FM-index.
Different lengths of the patterns are be investigated.
The amount of patterns is fixed, as we are interested in the throughput and average energy consumption.
However, the amount of patterns should be big enough to exploit parallelization on the FPGA.

For big texts, we will use the technique described in \ref{section:bwt_construction} to reduce the size of the rank matrix and suffix array.

In terms of hardware we will use the Xilinx U250 Data Center Accelerator Card as the FPGA.
The CPU is to be determined.

We expect that the FPGA implementation will have a higher throughput, as there is potential for high paralellization.
We also expect the FPGA implementation to be more energy efficient, as is usually the case.

[TODO: show which corpora we actually use and information about them]

[TODO: explain comparison]

[TODO: CPU hardware summary]
